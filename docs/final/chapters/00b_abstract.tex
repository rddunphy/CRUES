% !TEX root = ../report.tex

\begin{abstract}
\noindent
Co-operative mobile robotics is a growing field, as multiple autonomous
agents can coordinate to solve many tasks faster than a single robot.
However, most existing multi-agent systems rely on centralised control and
telemetric communication. This is a drawback in environments such as caves
or tunnels, where such techniques are restricted due to conditions which
limit the range of radio communication.

This project aimed to develop co-operative robots which map and search an
area and locate a goal using non-telemetric communication and distributed
control, allowing the system to perform in environments with restricted
telemetry. In order to simulate these conditions, communication over Wi-Fi
was artificially restricted to require line-of-sight. Each agent is able to
solve the problem individually; however, the introduction of additional
robots to the area should allow them to solve it faster by distributing the
task.

Three differential wheeled robots were constructed, with incremental encoders
and inertial measurement units used to track their odometry as
they traverse their environment. Computer vision is used together with an
array of ultrasonic sensors to allow the robots to detect objects, identify
other robots, and produce a map of their surroundings. The robots were
tested in a custom-made modular testing environment, and their performance
was evaluated both individually and co-operatively.

This report presents a detailed analysis of the electrical, mechanical,
and software design challenges as well as their solutions. Unforeseen setbacks
and technical difficulties delayed the completion of some of the more advanced
objectives, which resulted in the AI module not being fully integrated.
The system was, however, implemented to a high standard, and its modularity
and flexibility will allow future work to be built on the achievements of
this project.
\end{abstract}
