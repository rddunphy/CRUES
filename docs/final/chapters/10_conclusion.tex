% !TEX root = ../report.tex

\chapter{Conclusion}\label{conclusion}

As previously mentioned, using multiple co-operating robots is a useful strategy 
when attempting to complete tasks which can be parallelised, such as a exploration
of a large area. However in some environments such as underground or in areas of 
high interference, communication is impossible or severly restricted and as a result,
the robots would not be able to co-operate to complete their task. Furthermore,
usually systems of co-operating robots use a centralised control node to coordinate. 
Our system was designed so each robot was homogenous and operated without 
this central control unit. This introduced reduncency, important given the dangerous
environments they could be used in, which allows the remaining robots to complete
the task in the event of the failure of another. The robots were also designed to
strictly only be able to communicate with other robots when they had line-of-sight
with each other to simulate potentially restrictive environments.  

The mechanical design and implementation of each of the robots can be deemed as 
successful as our key requirements were met. To ensure the system was highly 
scalable the mechanical design had to be accurately reproduced. This would allow the 
software to be used to be across all the robots without any alterations needed. As a 
result, choosing to use a pre-built chassis was the correct decision. Although given 
the unforeseen issues with regards the reliability of the Pololu kit, the 
requirements to restrict the total cost means there likely is not a better 
alternative. Similarly, the mechanical design of the modular maze environment can
be deemed highly successful. It was very important that the robots could be 
tested easily on different maze configurations. Although it took longer than 
expected to produce the maze, it is of a very high quality, is extremely 
customisable and can be reused for any similar projects in the future.

Although issues arose in the process, the end products of the electric deisgn
and implementation were of a high standard. The decision to use the chassis'
accompanying power distrubtion board wasjustified as it fit both
mechanically and electronically with the other parts. This importantly allowed
other areas to be prioritised. Using ultrasonics was the best method of range 
detecting even if they are not perfect due to their wide cone of detection and
cost compared to infrared sensors. The 6-DOF IMU and encoders successfully
provided the neccessary odometry data to track the robot's movements around the
maze. The design of the PCB can also be deemed a success. Although multiple 
iterations of the PCB were designed to improve on the original design, the
thorough planning and design resulted in the first PCB stil being fully functional 
and not needing replaced. 

Ultimately, the decision to use ROS was justified due to it being able to handle
the overall architecture of the system with the publish/subscribe data model
whilst providing accompanying libraries that provide some of the required 
functionality. Although intergrating these libraries was not as simple as 
initially thought, they provided a greater level of functionality than if 
everything was implemented from scratch. The communication module was a great
success with the testing showing it to be extremely thread-safe and allows for
the robots to always being listening for incoming messages. The computer vision
node also works fully within the constraints of the maze. For us in more complex 
environments, a more detailed computer vision would be required.

In summmary, the project delivered the majority of the objectives intitially set
out where each individual components was designed, implemented and tested. Delays
and unforeseen complications resulted in the AI modules not being fully implemented. 
In addition, the complexity of SLAM results in this only being possible at lower 
resolutions, coupled with the uncertainty and timeouts from the ultrasonic sensors, 
the results for this were not as accurate as anticipated. However with greater 
processing power and more time, it is believed this would have been successfully 
implemented. Despite these issues, the success of each of the earlier components
such as the mechanical and electrical design, the communication software, odometry 
software and computer vision results in the project being overall very successful.