% !TEX root = ../report.tex

\chapter{Conclusion}\label{conclusion}
This project aimed to investigate co-operative robotics within 
communication restricted environments by constructing several robots, 
which used scalable methods to map and explore their surroundings and find 
a goal. The project was divided into three main aspects, each related to 
their own engineering discipline, namely: Mechanical, Electrical and 
Software. Research into these topics provided knowledge which allowed 
design decisions to be taken for each of the main aspects of the project. 
Agile methodologies were employed to manage the team over the course of 
the project and were proven successful with 5 out of 6 Major objectives 
completed. Influenced by the design decisions, a pre-built chassis was 
used as the basis for a mechanically functional robot using odometry, 
exteroreceptive sensors and actuators to explore its environment. 

These sensors included ultrasonic sensors, an IMU, encoders and a basic 
camera each of which were mounted on the robot via a bespoke PCB. The PCB 
design was iterated upon throughout the project to achieve a final PCB 
which connected each of the components in the most mechanically and 
electrically sound manner. 

In order for each of the components to function together, a software 
architecture based on data-oriented software design principles was 
created. This architecture centred on the Robot Operating System (ROS) 
library which uses a ``publish-subscribe'' model to allow modules in the 
system to communicate. For each of the aforementioned components of the 
system, a software module was written based on ROS node design principles. 
Each of these modules was independently tested extensively. 

To allow the robot to explore, ROS libraries for differential drive, 
sensor-fusion and Simultaneous Localisation and Mapping (SLAM) were used. 
Introducing each of these libraries added complexity to the project and 
issues were encountered throughout integration. Although a number of these 
issues were overcome, the delay in completing these objectives meant that 
the final objective --- pertaining to an AI control module --- could not 
be completed. 

As a consequence, system testing results obtained can only be analysed as 
benchmark results which provide a baseline for future work going forward. 
Despite this, overall the project had a myriad of successful outcomes and 
valuable lessons learned.      
\todo{Does need something else}

\todo{not a complete conclusion, two versions above and below}












Tasks that can be parallelised can be solved more 
effectively by using multiple co-operating robots, such as exploration of a large 
area. However, in some environments, such as underground or in areas of high 
interference, communication is severely restricted and the robots would not be able
to co-operate to complete their task. Furthermore, usually systems of co-operating robots 
use a centralised control node to coordinate. The system described was designed so each robot was 
homogeneous and operated without a central control unit. This introduced redundancy ---
an important consideration given the potentially dangerous environments of operation --- allowing 
the remaining robots to complete the task in the event of the failure of another. 
The robots were also designed to strictly only be able to communicate with other robots 
when they had line-of-sight with each other to simulate potentially restrictive 
environments.  

The mechanical design and implementation of each of the robots was 
successful as our key requirements were met. To ensure the system was  
scalable the mechanical design had to be reproducible to a high standard, thus allowing the software to be used to be across all robots without any alterations. As a 
result, choosing to use a pre-built chassis was the correct decision. Although given 
the unforeseen issues with regards to the reliability of the motors bundled in the Pololu kit, the 
requirements to restrict the total cost to allow scalability reduce possible alternatives. Similarly, the mechanical design of the modular maze environment was also successful. It allowed the system to be tested easily on
different maze configurations, which was essential for the project. The maze is of a very high quality, is highly 
customisable and can be reused for any similar projects in the future.

Although issues arose in the process, the end products of the electric 
and implementation were of a high standard. The decision to use the chassis'
accompanying power distribution board was justified as it fit both
mechanically and electronically with the other parts, allowing other areas to be 
prioritised. Using ultrasonic sensors was the best method of range detecting even if
they are not perfect due to their wide cone of detection and
cost compared to infrared sensors. The 6-DOF IMU and encoders successfully
provided the neccessary odometry data to track the robot's movements around the
maze. The design of the PCB can also be deemed a success. Although multiple 
iterations of the PCB were designed to improve on the original design, the
thorough planning and design resulted in the first PCB stil being fully functional 
and not needing replaced. 

Ultimately, the decision to use ROS was justified due to it being able to handle
the overall architecture of the system with the publish/subscribe data model
whilst providing accompanying libraries that provide some of the required 
functionality. Although intergrating these libraries was not as simple as 
initially thought, they provided a greater level of functionality than if 
everything was implemented from scratch. The communication module was a great
success with the testing showing it to be extremely thread-safe and allows for
the robots to always being listening for incoming messages. The computer vision
node also works fully within the constraints of the maze. For use in more complex 
environments, a more detailed computer vision would be required.

In summmary, the project delivered the majority of the objectives intitially set
out where each individual components was designed, implemented and tested. Delays
and unforeseen complications resulted in the AI modules not being fully implemented. 
In addition, the complexity of SLAM results in this only being possible at lower 
resolutions, coupled with the uncertainty and timeouts from the ultrasonic sensors, 
the results for this were not as accurate as anticipated. However with greater 
processing power and more time, it is believed this would have performed better.
Despite these issues, the success of each of the earlier components
such as the mechanical and electrical design, the communication software, odometry 
software and computer vision results in the project being overall very successful.