% !TEX root = ../report.tex

\chapter{Further Work}\label{furtherwork}
As not all of the major objectives of the project were completed, the main 
aspect of further work is to implement various AI algorithms and investigate 
their effectiveness in solving mazes co-operatively. A series of benchmark 
results have been provided in Section~\ref{systest/results} to evaluate 
these by. Various improvements could also be made to the SLAM (see Section~
\ref{soft/SLAM}) and Odometry (see Section~\ref{soft/odometry})\todo{change 
and change this ref if needed}. The exploration through testing of other 
libraries and methods of improving this system when using ultrasonic sensors 
and limited feedback methods, could take place and may provide improvements 
to the mapping capabilities of the robot. 

Subsequent to these improvement related aspects of further work, more robots 
could be added to the system in order to stress test the scalability of the 
project and its components such as ultrasonic synchronisation, 
communications and path-finding. If these stress tests were successful the 
maze environment could be altered to have walls and a floor which more 
closely simulate a real life environment. These changes may include adding 
texture to the walls to evaluate the response of the ultrasonic sensors; 
using the maze in a darkened environment to evaluate the response of the 
computer vision node and if additional lighting is required on the robots; 
and altering the texture of the floor to evaluate the response of the motor 
control system and the stability of the robots. 

Following these evaluations, the system could then either be upscaled in 
mechanical terms to perform in the field or used directly and tested in the 
field. Explorations of different real-life environments could be evaluated 
and the challenges which come with those tackled. This could lead the 
project to being a cheap, modular option which could be used in a variety of 
applications such as those described in this report. 