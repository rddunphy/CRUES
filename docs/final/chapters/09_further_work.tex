% !TEX root = ../report.tex

\chapter{Further Work}\label{furtherwork}
As not all of the major objectives of the project were completed, the primary objective going forward would be to implement various AI algorithms and investigate 
their effectiveness in solving mazes co-operatively.  A series of benchmark 
results have been provided in Section~\ref{systest/results} to evaluate 
these by. It is worth noting that A* is likely to be the best path planning 
algorithm once the maze has been fully or mostly explored. To 
explore the maze, many different algorithms could be used. Using periodic scans
to improve the SLAM map, a frontier-based algorithm could be built upon, using a 
depth first approach, to avoid doubling back. 
Robot behaviour on goal discovery would also have to be, explicitly: whether 
one goal being discovered was an end state --- meaning the robot could return 
to its starting point --- or whether to continue searching for more goals.
  
Various improvements could also be made to the SLAM (c.f. Section~
\ref{soft/SLAM}) and Odometry (c.f. Section~\ref{soft/odometry}).  
The exploration of other libraries to make improvements to the mapping 
capabilities of the robot, when using ultrasonic sensors and limited feedback 
methods, could be explored through testing their performances as part of
further work. Two such libraries are: SLAMOT (SLAM with Object Tracking) which 
could be used to resolve the issue of robots existing in each other's maps, as 
moving objects would be tracked and remain unmapped; and AMCL for
localisation, which compensates for the drift in time between the odometry and 
reference frames.   

The second optional objective, functionality for map sharing and SLAM loop 
closure across robots, could also be implemented. This would allow robots which 
have covered different paths to communicate their knowledge of the maps and 
improve search efficiency. It would also allow both robots to improve the 
accuracy of their existing map by combining it with that of the other robot.
To implement loop closure, reference points would need to be added
to the maze environment. These landmarks provide the robots a stationary point 
from which other measurements can be made relative to, increasing the accuracy 
of the maps created.

Subsequent to these, more robots could be added to the system in order to 
stress test the scalability of the project and its components. In areas such as
the ultrasonic range finders which are likely to produce bottlenecks in the system
with more agents, alternative or improved options should be explored. For example,
devising a method for synchronising ultrasonic pulses to prevent interference or
utilising more complex and expensive range finders such as LIDAR. 

Once these stress tests were successful, the maze environment could be altered to
have walls and a floor which more closely simulate a real-life environment. These
changes may include adding texture to the walls to evaluate the response of the
ultrasonic sensors; using the maze in a darkened environment to evaluate the
response of the computer vision node and if additional lighting is required on the
robots; and altering the texture of the floor to evaluate the response of the
motor control system and the stability of the robots. 

Following these evaluations, the system could then either be upscaled in 
mechanical terms to perform in the field or used directly and tested in the 
field. Explorations of different real-life environments could be evaluated 
and the challenges which come with those tackled. This could lead the 
project to be a cheap, modular option which could be used in a variety of 
applications such as those described in this report. 

Using a similar software and sensing system, underwater or aerial agents 
could also be created and tested, further evaluating the software system's 
effectiveness and the overall modularity of the system. 