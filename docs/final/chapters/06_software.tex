% !TEX root = ../report.tex

\chapter{Software}\label{software}

\todo{Bit in here about overall structure and architecture}



\section{ROS}\label{soft/ROS}

\subsection{Design}\label{soft/ROS/design}

\subsection{Implementation}\label{soft/ROS/impl}

\subsection{Testing}\label{soft/ROS/test}



\section{Communication}\label{soft/comms}
The communication node's requirements were to allow the agents to 
be able to send and receive messages to each other. The structure 
of these messages should allow for any object to be able to be sent 
regardless of the data types, depth or complexity of the objects. 
Each robot should also be able to simultaneously be listening for 
incoming messages whilst sending messages to a different robot. 
With the aim of allowing scalability, the communication system 
should be able to handle robots listening to multiple robots at the 
same time, whilst also being connected to send messages to different 
robots at the same time. 

The communication system also requires a network or communication 
technology to allow all of the robots in the system to be able to 
communicate with each other. Each of the robot's will also require 
a unique identifier to allow sending messages to specific robots 
rather than broadcasting all messages to all robots. 

\subsection{Design}\label{soft/comms/design}
In order to implement these requirements the first decision was the 
method the robots were going to use to communicate. A few options 
for this were considered such as Bluetooth and WiFi. After 
researching  and considering these options, we initially decided to 
use a Wireless Ad-hoc Network (WANET). 

WANET is a wireless network where the nodes can be located anywhere 
globally and does not require any infrastructure such as a router or 
mobile network. The underlying design is such that the nodes believe 
they are part of a single-hop or multiple-hop wireless network at the 
physical layer and the data link layer as part of the MAC sublayer~
\cite{rajesh2015congestion}. The wireless channels are often shared 
and so use carrier sense multiple access protocols to handle multiple 
nodes attempting to use the channel at the same time. This is known 
as link-level congestion and increases packet service time, 
decreasing utilization and overall throughput. Although scalability 
is an important  factor, it was deemed that WANET's handling of 
congestion was sufficient given the anticipated level of 
communication across the system at any time would be low. 

The WANET gives each node on the network its own local IP address. 
Anything that connects to the network can then send messages through 
the network provided it knows this address. As a result, a robot 
needs a way to find this address if it knows who it wishes to send 
a message to. The simplest method for this is to use a lookup table 
that maps the robot's name to their fixed IP in the network, allowing 
all nodes in the network access to this.

After deciding to use WANET, a decision  had to be made in how to format 
the data to send over the network. A few options were available such as 
using JSON, XML or a custom-structure. It was decided the best option was 
to use JavaScript Object Notation (JSON) objects after consultation with 
Dr~Irvine as discussed in Section~\ref{pm/consultations}. JSON is a 
standard data-interchange format that is easy for humans to read and write 
as it uses attribute-value pairs and array data types.

For the structure of the communication module after considering various options, 
a client-server system was decided upon. The client is the node 
responsible for sending the messages whilst the server is responsible 
for listening for incoming messages. As these would be separate nodes, 
these would work independently of each other and so would allow for the 
robot to send and receive messages simultaneously. 

\subsection{Implementation}\label{soft/comms/impl}
Look-up table
First copy of send/receive that caused multi-message issues
The new implementation with the thread-safe code

\subsection{Testing}\label{soft/comms/test}




\section{SLAM \& Sensor Fusion}\label{soft/SLAM}

\subsection{Design}\label{soft/SLAM/design}

\subsection{Implementation}\label{soft/SLAM/impl}

\subsection{Testing}\label{soft/SLAM/test}



\section{Computer Vision}\label{soft/cv}

\subsection{Design}\label{soft/cv/design}

\subsection{Implementation}\label{soft/cv/impl}

\subsection{Testing}\label{soft/cv/test}



\section{AI \& Control Modules}\label{soft/ai}

\subsection{Design}\label{soft/ai/design}

\subsection{Implementation}\label{soft/ai/impl}

\subsection{Testing}\label{soft/ai/test}